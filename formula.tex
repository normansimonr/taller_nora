\documentclass[12pt,letterpaper]{article}
\usepackage[utf8]{inputenc}
\usepackage{amsmath}
\usepackage{amsfonts}
\usepackage{amssymb}
\author{Norman Simón Rodríguez}
\begin{document}
La notación es la siguiente:
\begin{align*}
Y&=ingreso\\
X_1&=ocupado\\
X_2&=anioseduc\\
X_3&=salariounitario \, (promedio\, del\, mercado)\\
X_4&=conexiones\\
X_5&=sexo\\
X_6&=nivelingles\\
X_7&=estadosalud\\
X_8&=empresario\\
X_9&=marginpropS\\
X_{10}&=costoeduc \, (promedio\, del\, mercado)\\
X_{11}&=cursofinanzas
\end{align*}


La función ingreso es la siguiente (para $t>0$):
\begin{align*}
Y_t&=X_1 [ aX_2  X_3  (X_4+b) - cX_3X_5 + dX_6^3 + Y_{t-1}X_7^2] + eX_8  (X_4+b) + Y_{t-1}X_{9, t-1} - X_{10} \Delta X_2\\
X_9&=\frac{Y_t(1+gX_{11})}{h \max(Y_t)}
\end{align*}
\end{document}